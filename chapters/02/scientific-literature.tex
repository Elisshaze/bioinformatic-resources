\graphicspath{{chapters/02/images}}
\chapter{Scientific literature}

\section{Literature sources}
All bioinformatics works are based on literature.
Different sources of literature can be found.

	\subsection{Primary literature}
	Primary literature is defined as original materials.
	It is authored by researchers, contains original research data and is usually published in a peer-reviewed journal.
	Primary literature works can be:

	\begin{multicols}{2}
		\begin{itemize}
			\item Journal articles or conference proceedings, which are usually the first formal appearance of a result.
			\item Original articles: the original research conducted by the authors, including methods and resources used.
			\item Letters or communications: short reports of original research focused on an outstanding finding whose importance means that it will be of interest to scientists in other fields.
		\end{itemize}
	\end{multicols}

	\subsection{Secondary literature}
	Secondary literature is the summary or review of the theories and results of original scientific research.
	Secondary literature works can be:

	\begin{multicols}{4}
		\begin{itemize}
			\item Open letters.
			\item Comments.
			\item News.
			\item Reviews.
			\item Correspondence.
			\item Opinions.
			\item Protocols.
		\end{itemize}
	\end{multicols}

\section{Structure of a scientific article}
Scientific articles tend to have a well defined structure, composed, in order, of:

\begin{multicols}{3}
	\begin{enumerate}
		\item Title.
		\item Abstract.
		\item Keywords.
		\item Introduction or background.
		\item Methods or experiments.
		\item Results or analysis.
		\item Discussion.
		\item Conclusion.
		\item References or bibliography.
		\item Figures and tables.
		\item Supplementary materials.
	\end{enumerate}
\end{multicols}

\section{Impact measures}
An impact measure is used to define the goodness of a research or if it had a big impact in the community.

	\subsection{Impact of a journal}
	A measure of impact of a journal measure the impact of the publication of a journal.
	It can be measured in different ways:

	\begin{multicols}{2}
		\begin{itemize}
			\item Impact factor (IF): a measure that reflects the average number of citations of articles published in a science journal.
				It can be biased due to self-citations, journal-forced citations and it does not take into account negative citations.
				It is computed as:

				$$IF_y = \frac{Citations_y}{Publications_{y-1}+Publications_{y-2}}$$

			\item Journal of Citation reports JCR.
			\item Scimago Journal Rank SJR.
		\end{itemize}
	\end{multicols}

	\subsection{Personal impact}
	The personal impact measure the impact of a researcher.
	I can be measured as:

	\begin{multicols}{2}
		\begin{itemize}
			\item H-index: an index that attempts to measure both the productivity and the impact of the published work of a scientist or scholar.
				A scholar with an index of $h$ has published $h$ papers each of which has been cited by others at least $h$ times.
				It serves as an alternative to more traditional journal impact factor metrics in the evaluation of the impact of the work of a particular researcher.
			\item Web of Science WOS.
			\item Scoups.
			\item Google Scholar.
		\end{itemize}
	\end{multicols}

	\subsection{Peer review}
	Peer-reviewed articles are also called refereed articles.
	Peer review allows to:

	\begin{multicols}{2}
		\begin{itemize}
			\item Independently verify theories and assumptions.
			\item To screen for the works ethic.
			\item Asses appropriateness for publication.
			\item Check for transparency of research.
			\item Assess the quality of the research.
		\end{itemize}
	\end{multicols}

	Depending on the journal or publisher this process can takes from weeks to months.
