\graphicspath{{chapters/07/images/}}
\chapter{Network analysis}

\section{Introduction}

	\subsection{Network definition}
	A network is a series of interconnected components, systems or entities.
	They can be used to describe a large variety of physical or abstract phenomena.
	Nodes can represent different entities and arcs any kind of interaction.

	\subsection{Networks in system biology}
	Networks are a relevant part of system biology, in particular since the advent in systems biology. Genome and genomics, proteome and proteomics are represented by networks.
	The objectives of system biology are:

		\begin{itemize}
			\item Comprehension at the level of system, representing it using formalisms like networks.
			\item Analysis of individual components.
			\item Analysis of interactions.
			\item Analysis of potential emerging properties.
		\end{itemize}

Network (graph theory) allows us to recognize properties seen only at system level and not individual, need to capture the whole system dynamics.
In biology, networks are used to model Signal transduction networks, gene regulatory networks, protein-protein interaction.

\section{Graphs}

	\subsection{Definition}
	The analysis of a network structure should be done using appropriate mathematical methods.
	Graph theory is the tool able to extract information from the networks.
	A graph is a mathematical object defined by a set of nodes and arcs.
	It is denoted $G = (V, E)$, such that $V = \{v|v\ nodes\}$ and $E = \{(i,j)|i,j\in V\}$.

	\subsection{Magnitude of a graph}
	The magnitude of a graph is characterized by:

	\begin{multicols}{2}
		\begin{itemize}
			\item The number of nodes $|V|$ or order of $G$.
			\item The number of arcs $|E|$ or size of $G$.
		\end{itemize}
	\end{multicols}

	\subsection{Degree of a graph}
	The degree of a node in a graphs is the number of arcs that are incident with that node.
	In a direct graph the out degree is the number of arcs going out of a node, while the in degree is the number of arcs directed into a node.

	\subsection{Weighted graphs}
	A weighted graph is a graph which arcs have associated a weight, generally defined as a weighting function:

	$$w:E\rightarrow\mathbb{R}$$

	\subsection{Complete graphs}
	A complete graph is a direct or indirect graphs in which each pair of nodes is adjacent.
	If $(u,v)$ is an arc in $G$ then $v$ is adjacent to node $y$.
	So, for a complete graph:

	$$(u,v)\in E\forall u,v\in V$$

	\subsection{Paths}
	A path is a sequence of nodes $(v_1,v_2,\dots,v_n)$ such that:

	$$\{(v_1,v_2), (v_2, v_3), \dots, (v_{n-1}, v_n)\}\subseteq E$$

		\subsubsection{Simple paths}
		A simple path is a path without repeated nodes:

		$$P=\{(v_1,v_2), (v_2, v_3), \dots, (v_{n-1}, v_n)\}\subseteq E\land\forall v_i,v_j\in P, v_i\neq v_j$$

		\subsubsection{Cycles}
		A cycle is path such that:

		$$P=\{(v_1,v_2), (v_2, v_3), \dots, (v_{n-1}, v_n)\}\subseteq E\land\forall v_i,v_j\in P, v_i\neq v_j\land v_1=v_n$$

		A graph is called cyclic if it contains a cycle, otherwise it is called acyclic.

	\subsection{Bipartite graphs}
	A bipartite graph is an indirect graph $G = (V,E)$ such that:

	$$(u,v)\in E\Rightarrow  u\in V_1 \land v\in V_2\lor v\in V_1\land u\in V_2$$

	\subsection{Graphs connections}
	An indirect graphs is connected if each pair of nodes is connected by a path.

		\subsubsection{Weakly connected graphs}
		A directed graph is weakly connected if for each pair of nodes $(u,v)$ it exists a directed path from $u$ to $v$ or from $v$ to $u$.

		\subsubsection{Strongly connected graphs}
		A directed graph is strongly connected if a directed path between each pair of nodes exists.

		\subsubsection{Sparse graphs}
		A graph is sparse if:

		$$|E|\sim|V|$$

		\subsubsection{Dense graphs}
		A graph is dense if:

		$$|E|\sim|V|^2$$

	\subsection{Subgraphs}
	A graph $G' = (V', E')$ is a subgraph of $G = (V,E)$ if:

	$$V'\subseteq V \land E'\subseteq E$$

	\subsection{Trees}
	Trees are complete, acyclic graphs.
	A tree $ T$ spans $G = (V,E)$ if $T = (V,E')$ and $E'\subseteq E$.

		\subsubsection{Understanding if a graph is a tree}
		Let $G = (V,E)$ an indirected graph, then the following statements are equivalent:

		\begin{multicols}{2}
			\begin{itemize}
				\item $G$ is a tree.
				\item Each pair of nodes in $G$ is connected by a unique single path.
				\item $G$ is connected, but if a node is removed from $E$, the resulting graph is not connected.
				\item $G$ is connected and $|E| = |V|-1$.
				\item $G$ is acyclic and $|E| = |V|-1$.
				\item $G$ is acyclic but if an edge is added to $E$, the resulting graph contains a cycle.
			\end{itemize}
		\end{multicols}

	\subsection{Clique}
	A clique in an indirect graph $G = (V,E)$ is a subset $V'$ of the set of nodes $V$ such that for each two nodes in $V'$ it exists a unique arc that connects them.
	So the subgraph induced by $v'$ is complete.

	\subsection{Isomorphisms}
	An isomorphism between two graphs $G$ and $G'$ is a biunivocal correspondence $f:V(G)\rightarrow V(G')$ such that $u$ and $v$ in $G$ are adjacent if and only if $f(u)$ and $f(v)$ are adjacent in $G'$.

	$$G = (V,E)\land G = (V',E'), f:V \rightarrow V': (u, v)\in E\Leftrightarrow (f(u), f(v))\in E'$$

	If an isomorphism between two graphs can be built they are isomorphic.

		\subsubsection{Automorphism}
		An automorphism is an isomorphism on a graph onto itself.

		\subsubsection{Building an isomorphism}
		Building an isomorphism is an important problem in computer science with a complexity to be defined.
		Isomorphisms are useful when comparing the structure of two graphs.

	\subsection{Representing a graph}

		\subsubsection{Adjacency matrix}
		A graph can be represented by and adjacency matrix.
		Let $G = (V,E)$, then the adjacency matrix is a matrix $|V|\times|V|$ such that $A_{uv} = w$ if $(u,v)\in E$ and $A_{uv} = 0$ if $(u,v)\not\in E$.
		It grows quadratically with the number of nodes and each arc is represented two times, so it will be symmetric for indirect graphs.
		It is not efficient for sparse graphs.

		\subsubsection{Adjacency lists}
		The adjacency list of a graph $G=(V,E)$ is an array of lists.
		Each node has a list of the nodes to which is adjacent.
		The space is proportional to $|V|+|E|$ and each arch is represented two times.
		It is not efficient for dense graphs.

\section{Networks analysis}
Comparing two big networks is not as easy. For example, defining isomorphisms is simply not feasible.
A field of research focuses on network properties and works on three main abstraction levels: analysis of \textbf{single elements}, analysis at the level of \textbf{groups} and \textbf{global} analysis.

	\subsection{Analysis of single elements}
	In the analysis of single elements the more important nodes are identified.
	\textbf{Centrality measures} are a class of measures that indicate of the importance of a node in a graph or network.
	Different centrality measures are available, the main ones being \textbf{degree}, \textbf{closeness} and \textbf{betweenness} centrality.
	But other are also used, like the eigenvector centrality.

		\subsubsection{Degree centrality}
		Degree centrality is measured as:

		$$DC(n) = degree(n)$$

		Nodes with high $DC$ are defined as hubs and usually they have important roles in a network.
E.g., in protein-protein interaction, the failure (or absence) of a hub could produce a dramatic cascade effect.

		\subsubsection{Closeness centrality}
		Closeness centrality is measured as:

		$$CC(n) = \frac{|V|}{\sum\limits_{n'}d(n,n')}$$

		Where $d(n,n')$ is the length of the shortest path between $n$ and $n'$.
		Can be seen a the ratio between the number of nodes and a notion of t\textit{total distance} between the node and all the others, calculated as shortest path.

		Nodes with high closeness centrality can access quickly other nodes and have rapid cascade effects on other nodes.
		Closeness us to rank the nodes in order of importance.

		\subsubsection{Eccentricity centrality}
		Eccentricity centrality is computed as:

		$$C_s(n) = \frac{1}{\max\{d(u,n):u\in V\}}$$

		The eccentricity is a measure of the centrality index.

		Is calculated by computing the shortest path between the node $v$ and all other nodes in the graph and then considering the longest shortest path.
		Higher eccentricity means that the node is proximal to other nodes.

		\subsubsection{Betweenness centrality}
		Betweenness centrality is computed as:

		$$BC = \sum\frac{\sigma_{st}(n)}{\sigma_{st}}$$

		Where $\sigma_{st}$ is the number of shortest paths between $s$ and $t$.
		Nodes with high betweenness centrality can have greater control in the propagation of an effect in a whole network.

		\subsubsection{Subgraph centrality}
		Subgraph centrality is computed as:

		$$SC(n) = \sum\limits_{k=0}^\infty \frac{\mu_k(n)}{k!}$$

		It accounts for the participation of a node in all sub graphs of the networks.
		$\mu_k(v)$ is the number of closed walks of length $k$ starting and anding in node $v$.

		\subsubsection{Eigenvector centrality}
		Eigenvector centrality is computed as:

		$$EC(n) = \frac{1}{\lambda}\sum\limits_{t\in M(n)} EC(t) = \frac{1}{\lambda}\sum\limits_{t\in V} a_{v,t}EC(n)$$

		Where $\lambda$ is a constant, $a$ is the adjacency matrix and $M(v)$ are the neighbours of $v$.
		It can be written in vector notation as $A\vec{x} = \lambda\vec{x}$.
		It is a measure of the influence of a node in a network.
		The hogh-scoring nodes contribute more to the score of the node in question.
		A high eigenvector score means that a node is connected to many nodes who themselves have high scores.

	\subsection{Analysis of groups}
	During the analysis of groups groups or nodes are identified that have cohesion characteristics.
	A typical analysis is network clustering, which means diving the graph by its characteristics.
	First a similarity function between nodes is defined in terms of network topology.
	For example, in the enrichment analysis (gene ontology) the higher the overlap, the closer the terms
	Then a method to group the nodes in terms of their similarity is applied.

	\subsection{Analysis of network}
	During the analysis at the level of network topological properties that are global to the network are identified.

		\subsubsection{Clustering coefficient}
		The clustering coefficient measures the degree at which nodes in a graph tend to e connected.

			\paragraph{Local clustering coefficient}
			A local clustering coefficient indicates how much the neighbours of a node are distance from being a clique.
			The local clustering coefficient $LCC(n)$ of a node $n$ is given by the number of links between the members of $N(n)$, the neighbours of $n$, divided by the number of potential links between them:

			$$LCC(n) = \frac{2|\{(u,v)|u\land v\in N(n)\land (u,v)\in E\}|}{|N(n)|(|N(n)|-1)}$$

			For directed graphs the $2$ factor is eliminated.

			\paragraph{Global clustering coefficient}
			A global clustering coefficient $GCC$ is the mean of all $LCC$ computed across nodes in a graph and is called average clustering coefficient.
			The same measure can be calculated counting the number of closed triplets in the network divided by the total number of triples.
			$GGC=LCC$ coincide when using the weighted mean.
			It gives the intensity of the phenomena in the graph

		\subsubsection{Average diameter}
		The distance between two nodes is the least number of arcs that should be crossed to go from one node to another.
		The shortest path is the path that satisfies this criteria.
		The average diameter $AD$ of a graph is the average shortest path computed across all pair of nodes of a graph.

		\subsubsection{Degree distribution}
		Let $P(k)$ be the percentage of nodes with degree $k$ in a graph.
		The degree distribution is the distribution of $P(k)$ computed on all $k$.
		It can be defined as the probability of a node to have a degree $k$.
		Different distributions indicate different network topology, for example random networks or scale free networks.
		In scale free networks multiple hubs are present and is a hierarchical structure with an exponential degree distribution.

	\subsection{Small world effect}
	Small world networks have low $AD$ and high $GCC$.
	Comparing different graphs mean to combine their different properties like randomness, modularity and heterogeneity.
	In particular in a regular graph each node has the same number of neighbours.
	In a random graph there is low AD.
	In a small world graph $GCC$ tends to be similar to a regular graph or to a bigger random graph and AD is similar to a random graph.
