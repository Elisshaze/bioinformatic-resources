\chapter{Motif analysis}

\section{Introduction}

	\subsection{Definition}
	A DNA motif is a pattern of nucleotide sequences.
	They are usually associated to DNA-protein binding site and so to regulatory regions.
	They are a small pattern, usually between $5$ and $30bp$ that can recur many times in the genome and many times in the same gene.
	Motifs can be:

	\begin{multicols}{3}
		\begin{itemize}
			\item Standard.
			\item Palindromes.
			\item Gapped.
		\end{itemize}
	\end{multicols}

	\subsection{Functions}
	DNA motif functions include:

	\begin{multicols}{2}
		\begin{itemize}
			\item Sequence specific binding sites reached by transcription factors, nucleases and ribosomes.
			\item mRNA processing:

				\begin{itemize}
					\item Splicing: exonic splicing enhancer ESE.
					\item Editing: protospacer adjacent motif PAM, a DNA sequence that immediately follows the target DNA sequence of the Cas9 nuclease in the CRISPR system.
					\item Polyadenilation.
					\item Transcription termination.
				\end{itemize}

		\end{itemize}
	\end{multicols}

	\subsubsection{Degenerate motifs}
	Motifs in regulatory regions are often similar but variable: they are degenerate.
	Transcription factors are often pleiotropic, meaning that they regulate a lot of genes, but they need to be expressed at different levels.
	Degenerate motifs cause non-specific binding: a protein can bind genomic position different with respect to the one corresponding to the expected functional state.

\section{Motif search}
The objectives of motif search are to identify:

\begin{multicols}{3}
	\begin{itemize}
		\item Over-represented motifs in the genome.
		\item Motifs conserved in ortholog sequences.
		\item Sequences that can be candidates for transcription factor binding.
	\end{itemize}
\end{multicols}

Motifs can be represented as a consensus sequence or as profiles like positional matrices or HMMs.

	\subsection{Consensus sequence}
	A consensus sequence represents the result of multiple sequence alignments with the goal of finding recurrent motifs across the sequences.
	This sequence can be potentially different from all input sequence: it presents only the most conserved sequences for each position.
	It is built such that it minimizes the distance from each input sequence at each position.
	It can be also written following a IUPAC notation.

	\subsection{Positional matrix}
	A positional matrix is an alternative way to represent a motif than the consensus sequencs.
	The elements in the matrix represent all possible bases at each position.
	Example of these matrices are:

	\begin{multicols}{2}
		\begin{itemize}
			\item Position frequency matrix PFM or PSWM.
			\item Position probability matrix PPM or PFM.
			\item Position weight matrix PWM or PSSM.
		\end{itemize}
	\end{multicols}

		\subsubsection{Populating a position frequency matrix}
		A position frequency matrix is computed as:

		$$M_{k,j} = \sum\limits_{i=1}^N\delta(X_{i,j} = k)$$

		Given $k$ the set of all symbols in the alphabet.
		$N$ the number of aligned sequences.
		$j$ iterates over the length of the sequence.

		\subsubsection{Populating a position probability matrix}
		A position probability matrix is very similar with respect to the position frequency matrix, with the exception that each cell represent the probability that in that sequence position a particular base will be found.
		Its cells are computed as:

		$$M_{k,j} = \frac{1}{N}\sum\limits_{i=1}^N\delta(X_{i,j} = k)$$

		\subsubsection{Assessing the probability that a sequence belong to a PPM}
		To assess the probability for a sequence to belong to a PPM the probabilities for each base $i$ found at each position $j$ are multiplied:

		$$P(seq\in PPM) = \prod\limits_{j=1}^R M_{seq_j, j}$$

		\subsubsection{Correcting PPMs}

			\paragraph{Laplace smoothing}
			Laplace smoothing introduces pseudocounts to allow to estimate probabilities in case of too few observations.
			A pseudocount is an amount added to the number of observed cases in order to change the expected probability.

			\paragraph{Adding a background model}
			Another way to correct PPMs is to add a background model.
			With this a new matrix is computed as:

			$$M_{k,j} = \log_2\frac{M_{k,j}}{b_k}$$

			Where $b$ represent a background model and can vary across nucleotides for organisms with high $GC$ content.
			It is typically computed as:

			$$b_k = \frac{1}{|k|}$$

			And so is $0.25$ for nucleotides and $0.05$ for amino acids.


	\subsection{Hidden Markov model}
	A Markov chain is a mathematical system that experiences transitions from one state to another according to certain probabilistic rules.
	The possible future states are fixed and not based on how the process arrived at its present state.
	The state is directly visible to the observer: state transition probabilities are the only parameter.
	The state remain transparent, while the output is easily obtainable.
	A HMM of the first order is defined as:

	\begin{multicols}{2}
		\begin{itemize}
			\item A finite set of states $S$.
			\item A discrete alphabet of symbols.
			\item A matrix of transition probabilities $T = P(i|j)$, the probability of transition from state $j$ to $i$>
			\item A matrix of emission probabilities (the probability of skipping a state) $T = P(X|i)$, the probability of $X$ emission in state $i$.
		\end{itemize}
	\end{multicols}

		\subsubsection{Assessing the probability that a sequence is generated by a HMM}
		The probability that a sequence is generated by a HMM can be computed as:

		$$P(S|w) = \sum\limits_\pi P(S, \pi |w)$$

		Where $S$ is the sequence, $w$ are the probabilities parameters, $\pi$ are all possible computationally inefficient paths.
		Efficient algorithm to compute this probability are Forward-Backward and Viterbi.

	\subsection{Sequence logos}
	Sequence logos are visual representation of positional matrices and simple HMM profiles.
	The height of each character in a sequence logo is proportional to its information content: $2$ bit if $1$ base occurs in all input sequences, $1$ if two bases occur and $0$ if all bases occur equally.
	The higher the variability, the lower the height of a specific base.
	In particular the height of base $b$ at position $l$ is computed as:

	$$f(b,l)R_{sequence}(l)$$

	Where:

	$$R_{sequence}(l) = 2-(H(l) + e(n))$$

	Such that $H(l)$ is the Shannon entropy and is computed as:

	$$H(l) = -\sum\limits_{b=a}^tf(b,l)\log_2 f(b,l)$$

	And

	$$e(n) = \frac{1}{\ln 2}\cdot \frac{4-1}{2n}$$

\section{Motif identification}
There are two types of motif identification: pattern matching and pattern discovery.

	\subsection{Finding known motifs - pattern matching}
	Pattern matching is the problem of finding known motifs, for example seeing if a binding of a protein to an upstream region of a gene is significant.
	In order to find out whether a transcription factor matches a promoter the PFM matrix is used to compute a score for each sliding window.
	This scores can be plotted against a threshold, so as to identify regions able to support a putative binding.

		\subsubsection{Total binding affinity}
		Total binding affinity TBA is a cutoff-free method.
		The TBA is a method used to describe the affinity of a DNA sequence for a transcription factor described by a PFM with a single score.
		It takes into account binding sites of all possible affinities and considers the whole sequence, keeping into account both high and low affinity sites.

	\subsection{Finding de novo motifs - pattern discovery}
	Pattern discovery if the problem of finding de novo motif, for example finding the motifs upstream of a specific gene.
	Given a set of sequences, the objective is to find the most represented motifs.
	Using the MEME suit it is possible to identify new sequences and through Jaspar they can be compared to already characterized transcription factors.
	Methods can be:

	\begin{multicols}{2}
		\begin{itemize}
			\item Exact: give optimal solution given specific parameters.
			\item Approximated: give suboptimal solution decreasing the computational burden.
				They are MULTIPROFILER, CONSENSUS, MEME, Gibbs sampler and MotifSampler for example.
		\end{itemize}
	\end{multicols}

		\subsubsection{Distance between a real motif and the consensus}
		The distance between a real motif and the consensus is generally less that that for two real motifs.
		The consensus sequence must be guess and a scoring function to compare different guesses and choose the best one must be chosen.

		\subsubsection{Elements of the problem}
		The problem of finding de novo motifs can be formalized considering the following elements:

		\begin{multicols}{2}
			\begin{itemize}
				\item $n$ the length of each sequence.
				\item $DNA$, an array of size $t\times n$.
				\item $l$, the length of the motif or $l$-mer.
				\item $s_i$, the starting position of an $l$-mer in sequence $l$.
				\item $s = (s_1, s_2, \dots, s_t)$, an array of motifs starting position.
			\end{itemize}
		\end{multicols}

		If the starting positions $s$ are given, finding the consensus is easy.
		When those are not given, finding the best motif is solving the median string problem.

		\subsubsection{The median string problem}
		Given a set of $t$ DNA sequences the objective is to find a pattern that appears in all $t$ sequences with the minimum number of mutations.
		The Hamming distance is used, such that:

		$$d_h(v, w) = \#\ nucleotide\ pairs\ that\ do\ not match\ when\ v\ and\ w\ are\ aligned$$

		Then, for each DNA sequence $I$, all $d_h(v,x)$ are computed, where $x$ is an $l$-mer with starting position $s_i$.
		Then the minimum $d_h(v,x)$ among all $l$-mers of the sequence.
		The $TotalDistance(v,DNA)$ is the sum of the minimum Hamming distances for each DNA sequence $I$, so

		$$TotalDistance(v,DNA) = \min\limits_s d_h(v,s)$$

		Where $s$ is the set of starting positions.
