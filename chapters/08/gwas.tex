\chapter{Genome wide association studies}

\section{Types of studies}

	\subsection{Descriptive studies}
	Descriptive studies are studies in which an hypothesis is generated.
	Then the patterns of disease occurrence in relation to variables such as person, place and time are studies.
	They are often the first step or initial enquiry into a new topic event, disease or condition.
	They typically estimate the frequency and the magnitude of the event analysed.

	\subsection{Analytical studies}
	An analytical study is one in which action will be taken on a cause system to improve the future performance of the system of interest.
	The focus is to test an hypothesis to produce predictive data.
	In particular they are used to identify factors that are associated with a disease or to quantify the risk of these factors.

	\subsection{Cohort studies}
	Cohort studies are a type of analytical studies that involve a cohort.
	A cohort is a well-defined group of individuals who share a common characteristic or experience.
	For example individual exposed to a drug, vaccine and pollutant.

		\subsubsection{Prospective cohort studies}
		Prospective cohort studies potential exposure has already occurred while outcomes have yet to occur.
		Participants are grouped according to past or current exposure and a follow-up in the future determine whether the predicted outcome occurs.

		\subsubsection{Retrospective cohort studies}
		Retrospective cohort studies both exposure and outcomes have already occurred.
		Participants are grouped according to past exposure and certain characteristics and are compared for a particular outcome.

		\subsubsection{Measures of associations}
		In cohort study a $2\times 2$ table can be built to determine, for example, the effect of exposure to a certain event on disease presence.
		This type of table is described on table \ref{tab:risk_table}.

		\begin{table}[H]
			\centering
			\begin{tabular}{ccc}
				 & \multicolumn{2}{c}{Disease}\\
				 & Yes & No \\
				 \cline{2-3}
				 Exposed & \multicolumn{1}{|c|}{$a$} & \multicolumn{1}{|c|}{$b$}\\
				 \cline{2-3}
				 Not exposed & \multicolumn{1}{|c|}{$c$} & \multicolumn{1}{|c|}{$d$}\\
				 \cline{2-3}
			\end{tabular}
			\caption{Cohort study's table example}
			\label{tab:risk_table}
		\end{table}

		Then from this table the percentage of individuals exposed harbouring the disease can be computed:

		$$I_e = \frac{a}{a+b}$$

		Also the percentage of individuals not exposed and not harbouring the disease:

		$$I_{ne} = \frac{c}{c+d}$$

		From this two measure the risk excess $RE$ and the relative risk $RR$ can be computed:

		$$RE = I_e - I_{ne} \qquad\qquad RR= \frac{I_e}{I_{ne}}$$

		The risk excess determine how the exposure betters, or worsen the chance of presenting a disease.
		The relative risk instead determine the nature of the exposure's event:

		\begin{multicols}{3}
			\begin{itemize}
				\item $RR<1$ indicates a protective factor.
				\item $RR \sim 1$ indicates an absence of risk.
				\item $RR> 1$ indicates a risk factor.
			\end{itemize}
		\end{multicols}

			\paragraph{Statistical confidence}
			From the table also a significance of association and a precision of association can be determined.
			The first requires a $p$-value, that determine how unlikely it is that the events observed arise by chance.
			The second requires a confidence interval $CI$, which can be with a certain level of confidence $\alpha$, usually around $80\%$ and $95\%$, that determine how many times the correct value will be inside the interval when the measurement or the estimate will be replicated many times.
			Usually the relative risk indicates the amount of random error around the point estimate.

	\subsection{Case-control study}
	The purpose of a case-control study is typically to study rare diseases or multiple exposures that may be related to a single outcome.
	Participants are selected based on outcome status:

	\begin{multicols}{2}
		\begin{itemize}
			\item Case subjects have outcome of interest.
			\item Control subjects do not have outcome of interest.
		\end{itemize}
	\end{multicols}

	This type of study is usually preferred when funding is limited.

		\subsubsection{Measure of association}
		The same table as a cohort study is built (as in table \ref{tab:risk_table}), but the measure of association is different.
		First the odds of exposure in cases is computed:

		$$\frac{\frac{a}{a+c}}{\frac{c}{a+c}} = \frac{a}{c}$$

		Then the odds of exposure in control:

		$$\frac{\frac{b}{b+d}}{\frac{d}{d+b}} = \frac{b}{d}$$

		Finally the ratio of this two measure is the odds ratio $OR$, the measure of association for a case control study:

		$$OR = \frac{\frac{a}{c}}{\frac{b}{d}} = \frac{ad}{bc}$$

		$OR$ is in relationship with $RR$ following the equation:

		$$OR = \frac{RR(1-R_0)}{1-RR\cdot R_0}$$

		Where $R_0$ is the frequency of the disease in the not exposed population.

\section{GWAS}

	\subsection{Objective of GWAS}
	The objective of a GWAS is to find connections between a phenotype known to be heritable and whole-genome genotype.
	GWAS were developed in $2004$, mainly thanks to the HapMap project, which unravelled the existence of linkage disequilibrium blocks, which allowed the exploitation of tag SNPs.
	Specific goals are distinct:

	\begin{multicols}{2}
		\begin{itemize}
			\item Identification of statistical connections between points or areas in the genome and the phenotype.
				The hypotheses are driven for biological studies of specific genes or regions in specific contexts.
			\item Generation of insights on genetic architecture or phenotype.
				In fact a phenotype could be due to many small genetic effects dispersed across the genome or due to few large effects concentrated in one area.
				An example in the second case is the MHC or major histocompatibility complex, a group of genes involved in the mechanism of immune defence.
			\item Build statistical models to predict phenotype from genotype.
		\end{itemize}
	\end{multicols}

	\subsection{Main application of GWAS}
	The main application of GWAS are:

	\begin{multicols}{2}
		\begin{itemize}
			\item Identification of susceptibility variants for novel biological insights like:

				\begin{itemize}
					\item Therapeutic targets.
					\item Biomarkers.
				\end{itemize}

			\item Improved measures of individual aetiological (the manner of causation of a disease or condition) processes for personalized medicine.
		\end{itemize}
	\end{multicols}

	\subsection{GWAS methodology}
	A typical GWAS methodology can be described as:

	\begin{multicols}{2}
		\begin{itemize}
			\item Collect $n$ subjects with known phenotype, usually $n\in [10^3;10^4]$.
			\item Measure each one in $m$ genomic locations representing common variation in the whole genome.
				Typically these are SNPs.
				Usually $m\in [10^5;10^6]$, but recently with whole genome sequencing $m = 3\cdot 10^9$.
			\item The data can be thought as a matrix $X$ of dimension $n\times m$ with subject as rows and SNPs as columns.
				This matrix is built such that $X_{ij} \in \{0,1,2\}$, representing the genotype at a single column.
				Moreover a vector of phenotypes $Y_n$ can be given.
		\end{itemize}
	\end{multicols}

	Having collected the data and having computed the matrix $X$ and the vector $Y$ the first task is association testing: finding SNPs (column of $X$) that are statistically associated with $Y$.
	This can be thought of as $m$ separate statistical tests run on the matrix $X$.

	\subsection{Single nucleotide polymorphisms}
	A SNP is defined as a single base variation in a DNA sequence.
	They are classified according to the minor allele frequency $MAF$:

	\begin{multicols}{2}
		\begin{itemize}
			\item Common SNPs have $MAF \ge 1\%$.
			\item Rare SNPs have $MAF < 1\%$.
		\end{itemize}
	\end{multicols}

		\subsubsection{SNPs frequency}
		In the human genome SNPs compose the $0.1\%$ and are what makes human unique.
		These variants can be:

		\begin{multicols}{3}
			\begin{itemize}
				\item Harmless.
				\item Harmful.
				\item Latent.
			\end{itemize}
		\end{multicols}

		They can lie in coding regions, but the majority of them are found in non-coding one.
		Thy are present between in $1$ every $1000$ bases or $1$ every $100$-$300$.
		The abundance of SNPs and the ease with which they can be measured make them very important.
		Two thirds of SNPs modification are from a $C$ to a $T$.
		They are typically found in non-coding regions and are found less in less conserved regions.
		In coding regions synonymous SNPs (that don't change the structure of the coded protein) are more common.

		\subsubsection{SNPs effects}
		SNPs can have different effect on the genome:

		\begin{multicols}{2}
			\begin{itemize}
				\item When they are found near a gene they can act as marker for that gene.
				\item SNPs in regulatory regions can modify transcription influencing the binding of transcription factors.
				\item SNPs in coding regions can modify the stricture of codified protein.
			\end{itemize}
		\end{multicols}

		\subsubsection{Nucleotide diversity}
		Nucleotide diversity measures the degree of polymorphism in DNA sequences or in a population.
		It is defined as the average number of nucleotide differences per site between two DNA sequences in all possible pairs in the same population and is denoted by $\pi$
		It is estimated as:

		$$\hat{\pi} = \frac{n}{n-1}\sum\limits_{ij} x_ix_j\pi_{ij} = \frac{n}{n-1} \sum\limits_{i=2}^n\sum\limits_{j=1}^{i=1}2x_ix_j\pi_{ij}$$

		Where:

		\begin{multicols}{2}
			\begin{itemize}
				\item $x_i$ and $x_j$ are the respective frequencies of the $ith$ and $jth$ sequences.
				\item	$\pi_{ij}$ is the number of nucleotide differences per nucleotide site between the $ith$ and $jth$ sequences.
				\item $n$ is the number of sequences in the sample.
				\item $\frac{n}{n-1}$ is a normalization factor that makes the estimator independent on how many sequences are sampled.
			\end{itemize}
		\end{multicols}

			\paragraph{Hardy-Weinberg equilibrium}
			In a population with genotypes $BB$, $bb$ and $Bb$, if:

			\begin{multicols}{2}
				\begin{itemize}
					\item $p = freq(B)$.
					\item $q = freq(b)$.
				\end{itemize}
			\end{multicols}

			The frequencies of the genotypes are then:

			\begin{multicols}{3}
				\begin{itemize}
					\item $freq(BB) = p^2$.
					\item $freq(bb) = q^2$.
					\item $freq(Bb) = 2pq$
				\end{itemize}
			\end{multicols}

			In a condition of equilibrium and will not change considering:

			\begin{multicols}{3}
				\begin{itemize}
					\item No mutations.
					\item No emigrations.
					\item Population of infinite size.
					\item No selective pressure.
					\item Random coupling.
				\end{itemize}
			\end{multicols}

		\subsubsection{Linkage disequilibrium}
		Two genetic loci are said to be in linkage disequilibrium $LD$ when there is a non-random association of alleles at different loci in a given population.
		It usually indicates that two alleles are near and in mammalians $LD$ is usually lost at around $100Kbp$.
		Let:

		\begin{multicols}{2}
			\begin{itemize}
				\item $p_A$ be the frequency of an allele $A$ in a genomic locus.
				\item $p_B$ be the frequency of an allele $B$ in another genomic locus.
			\end{itemize}
		\end{multicols}

		The association between allele $A$ and allele $B$ is random when:

		$$p_{AB} = p_Ap_B$$

			\paragraph{Measuring linkage disequilibrium}
			The coefficient $D$ is a measure of linkage disequilibrium.
			It is defined for two biallelic loci with alleles $A$ and $a$ at the first locus and $B$ and $b$ at the second one as:

			$$D_{AB} = p_{AB} - p_Ap_B\qquad\qquad D_{Ab} = -D_{AB}\qquad\qquad D_{ab} = D_{AB}$$

			Being $LD$ a property of two loci and not of their alleles, it is the magnitude being of interest, not the sign.
			The magnitude does not depend on the choice of the allele, and the range of $D$ changes with allele frequency.
			Knowing that $p_{AB}$ is smaller than $p_A$ and $p_B$ and that the frequencies cannot be negative:

			$$-p_Ap_B\land -p_ap_b\le D_{AB}\le p_ap_B\land p_Ap_b$$

			The possible values of $D$ depend on the allele frequencies and as such is difficult to interpret.
			Because of this it is normalized in $D'$:

			$$D'_{AB} = \begin{cases}\frac{D_{AB}}{\max(-p_Ap_B, -p_ap_b)} & D_{AB} < 0\\\frac{D_{AB}}{\min(p_ap_B, p_ap_b)} & D_{AB}>0\end{cases}$$

				\subparagraph{Measuring LD with $r^2$}
				To measure $LD$ with $r^2$ two random variables are defined:

				\begin{multicols}{2}
					\begin{itemize}
						\item $X_A$ such that $X_A=1$ if allele at locus $1$ is $A$ and $X_A=0$ if the allele is $a$.
						\item $X_B$ such that $X_B=1$ if allele at locus $2$ is $B$ and $X_B=0$ if the allele is $b$.
					\end{itemize}
				\end{multicols}

				Or:

				$$X_A = \begin{cases}1 & allele=A\\0 & allele = a\end{cases}\qquad\qquad X_B = \begin{cases}1 & allele = B\\0 & allele = b\end{cases}$$

				Then the correlation between the two random variables can be defined as:

				$$r_{AB} = \frac{Cov(X_A, X_B)}{\sqrt{Var(X_A)Var(X_B)}} = \frac{D_{AB}}{\sqrt{p_A(1-p_A)p_B(1-p_B)}}$$

				And:

				$$r^2_{AB} = \frac{D^2_{AB}}{p_A(1-p_A)p_B(1-p_B)}$$

				This measure is usually employed as it is always a positive value.

				\subparagraph{Classifying LD}
				$LD$ can be classified according to the $D'$ and $r^2$ values:

				\begin{multicols}{2}
					\begin{itemize}
						\item When $D'= 1$ there is complete $LD$.
						\item When $r^2 = 1$ there is perfect $LD$.
					\end{itemize}
				\end{multicols}

				Perfect $LD$ implies complete $LD$.
				There are situations in which $D'=1$ and $r^2$ is low, so usually both measures are reported.

			\paragraph{Haplotypes}
			An haplotype is a set of linked SNPs on the same chromosome.
			Genotypes don't report informations about the connections of alleles at different SNPs loci, so there could be several possible haplotypes for the same genotype.
			An haplotype block is defined as a cluster of SNPs in linkage disequilibrium and an haplotype boundary as sequences of blocks with strong internal linkage disequilibrium but no linkage disequilibrium between them.
			They usually reflect genetic recombination hotspots.

			\paragraph{Tag SNPs}
			Tag SNPs are a set of SNPs that captures most variations in haplotypes, removing redundancy.
