\graphicspath{{chapters/03/images}}
\chapter{Biological databases}

\section{Introduction}

	\subsection{Classification of databases}
	A huge number of biological databases are available and they can be distinguished as:

	\begin{multicols}{2}
		\begin{itemize}
			\item Primary databases containing sequences of nucleotides and amino acids.
			\item Derived and specialized databases containing protein domains and motifs, protein structures, genes, transcripts, expression profiles, variations, pathways and many other informations.
		\end{itemize}
	\end{multicols}

	Each database is characterized by a central biological element which constitutes the object around which the principal entry of the database is constructed.

	\subsection{Data sources}
	Data in these databases is derived from:

	\begin{multicols}{3}
		\begin{itemize}
			\item Literature.
			\item In-vitro and in-vivo analysis.
			\item In-silico analysis.
		\end{itemize}
	\end{multicols}

	\subsection{Nomenclature}
	One of the main problems related with biological databases is nomenclature.
	There can be different name for the same gene or different genes with the same name.
	To uniquely identify genes and proteins and manage the large amount of information related, primary data banks assign an accession number to each element they store.

	\subsection{Reference genome}
	A reference genome is a digital sequence of nucleic acids assembled to be a representative sequence for a given species.
	It is assembled from DNA sequencing of a set of donors.
	An example of reference genome is \emph{GRCh38} from which \emph{hg38} is derived aggregating many donor informations.

\section{Popular databases}

	\subsection{GenBank}
	GenBank contains nucleotide sequences.
	The aim of the database is to store and archive historically important but redundant nucleotide sequences.
	Data can be submitted singularly or in a batch manner.

	\subsection{RefSeq}
	RefSeq is a curated and non redundant collection of DNA, RNA and protein sequences.
	Each RefSeq entry represents a sings molecule in a particular organism.
	Its basis is compiled with a process of collaboration, extraction and computation from GenBank.
	Each molecule is annotated reporting the name of the organism, the correct gene symbol for that organism and informative names of proteins when possible.

	\subsection{UniProt}
	UniProt is a comprehensive, high-quality and freely accessible resource of protein sequence and functional information.
	It provides protein sequences, domains and structural information like subcellular location for many species.
	It also includes some alignment and mapping tools.

	\subsection{Others}
	Other examples of derived databases are:

	\begin{multicols}{4}
		\begin{itemize}
			\item dbGap.
			\item Structure.
			\item Gene.
			\item Biosystems.
		\end{itemize}
	\end{multicols}

	\subsection{Genome browser}
	A genome browser is a database containing reference sequence assemblies for one or more genomes.
	It allows to browse data at various detail levels, from chromosome to gene, down to a single exon or intron.
	It also allows for the comparison between species and data extraction.

		\subsubsection{UCSC genome browser}
		The UCSC genome browser contains the genome of about $100$ species, but it does not provide a browser for all of them.
		It integrates informations like SNPs, sequence conservation, regulatory elements (ENCODE) and others.

			\paragraph{UCSC table browser}
			The UCSC table browser allow to extract data from the database tables without the need for a graphical interface.
			It can also align sequences, annotate SNPs and convert data between genome versions.
			It is a flexible tool that can retrieve data for one or more genes in a variety of formats.
			When submitting heavy task it will redirect them to Galaxy, an online workflow system.

		\subsubsection{Ensembl genome browser}
		The Ensembl genome browser is the European genome browser.
		It focuses on vertebrate genomes.
		It includes genomic variants, both somatic and structural, and regulatory elements data.
		It offers an interface to access data directly BioMart.
		All Ensemble transcripts are based on proteins and mRNAs contained in the databases:

		\begin{multicols}{3}
			\begin{itemize}
				\item UniProt/Swiss-Prot (manually curated).
				\item UniProt/TrEMBL (not reviewed).
				\item NCBI RefSeq (manually curated).
			\end{itemize}
		\end{multicols}

			\paragraph{Biomart}
			Biomart is a data mining platform which is able to address complex queries on ENSEMBL.
			It is similar to the USCS table browser, while being more powerful as it can retrieve both annotation and sequences.
